\documentclass[a4paper]{article}
\usepackage[slovene]{babel}
\usepackage[T1]{fontenc}
\usepackage[utf8]{inputenc}
\usepackage{lmodern}
\usepackage{amsfonts}
\usepackage{amsmath}
\usepackage{makeidx}
\usepackage{siunitx}
\usepackage{graphicx}
\usepackage{subfigure}
\usepackage{amsfonts}
\usepackage{amssymb}


\title{Seminar \\\vspace{2cm} {\huge Panjerjev razred}\vspace{2cm}}
\author{Lana Herman \\[1.5mm] Tin Markon \\[1.5mm]\vspace{7cm}
Mentor: prof. dr. Janez Bernik \\
Univerza v Ljubljani \\[1.5mm]
Fakulteta za matematiko in fiziko \vspace{3cm}}
\date{Marec, 2019}

\begin{document}

\begin{titlepage}
\clearpage \maketitle
\thispagestyle{empty}
\end{titlepage} 


\tableofcontents
\pagebreak

\maketitle
\section{Problem}
\textit{Za verjetnostno masno funkcijo} $p : \mathbb{N}_{0} \rightarrow [0, 1]$ \textit{slučajne spremenljivke z vrednostmi v $\mathbb{N}_{0}$ pravimo, da je v Panjerjevem razredu, če obstajata realni števili a in b taki da je}
$$p_k = p_{k-1}(a + \frac{b}{k}), \hspace{0.25cm}  k \in \mathbb{N}.$$
\textit{Dokaži, da je p v Panjerjevem razredu, če in samo če je ene izmed sledečih oblik (povsod teče n po $\mathbb{N}_{0}$):}
\begin{enumerate}
	\item $p_n = \delta_{0}(n)$ \textit{(Diracova masa v 0).}
	\item $p_n = \frac{\lambda^n}{n!}e^{-\lambda}$ \textit{za nek $\lambda \in (0, \infty)$ (Poissonova porazdelitev).}
	\item $p_n = {{\alpha + n - 1}\choose n}p^n(1-p)^{\alpha}$ \textit{za neka $\alpha \in (0, \infty), p \in (0,1) \\ $(Tu je ${{\alpha + n - 1}\choose n}$ posplošeni binomski simbol, ki se izraža ${{\alpha + n - 1}\choose n}=\frac{\Gamma(\alpha + n)}{\Gamma(n+1)\Gamma(\alpha)}$.}
	\item $p_n = {N \choose n}p^n(1-p)^{N-n}$ \textit{za neka $p \in (0,1), N \in \mathbb{N}$ (binomska porazdelitev).}
\end{enumerate}
\vspace{0.5cm}
\section{Rešitev}
$(\Leftarrow)$ Najprej predpostavimo, da so Diracova masa v 0, Poissonova porazdelitev, negativna binomska porazdelitev ter binosmak porazdelitev v Panjerjevem razredu. Iščemo taki realni števili a in b, da bo veljala enakost $p_k = p_{k-1}(a + \frac{b}{k}).$
\begin{enumerate}
	\item{$p_1=p_0(a + b) = 0 = p_2 = p_3 = \ldots = p_k, \hspace{0.25cm} \forall k \in \mathbb{N},$ sledi $a + b = 0, a=-b$}
	\item{$\frac{p_k}{p_{k-1}}=\frac{\lambda^k e^{-\lambda}(k-1)!}{\lambda^{k-1} e^{-\lambda}k!} = \frac{\lambda}{k}, $ sledi $ a = 0, b = \lambda$} 
	\item{Ker velja $\Gamma(n)=(n-1)!$  za $\forall n \in \mathbb{N}$, in $\Gamma(\alpha+1)=\alpha\Gamma(\alpha)$ za $\forall \alpha > 0,$ velja \\[1.5mm] ${n +\alpha - 1 \choose n} = \frac{(n+\alpha-1)!}{n!(\alpha-1)!}=\frac{\Gamma(n+\alpha)}{n!\Gamma(\alpha)}.$ \\[1.5mm] $\frac{p_n}{p_{n-1}}=\frac{\binom{n + \alpha - 1}{n}p^n(1-p)^{\alpha}}{\binom{n-1+\alpha-1}{n-1}p^{n-1}(1-p)^\alpha}=\ldots=\frac{\Gamma(\alpha+n)}{n\Gamma(\alpha+n-1)}p=\frac{(\alpha+n-1)\Gamma(\alpha+n-1)}{n\Gamma(\alpha+n-1)}p= \\[1.5mm] p+p(\alpha-1)\frac{1}{n}, $ sledi $ a=p, b=p(\alpha-1)$}
	\item{$\frac{p_n}{p_{n-1}}=\frac{\binom{N}{n}p^n(1-p)^{N-n}}{\binom{N}{n-1}p^{n-1}(1-p)^{N-n+1}}=\ldots=\frac{p}{1-p}(N+1)\frac{1}{n}-\frac{p}{1-p}, $ sledi $ a= \\[1.5mm] -\frac{p}{1-p}, b = \frac{p}{1-p}(N+1)$}
\end{enumerate}

\pagebreak
$(\Rightarrow)$ Sedaj predpostavimo, da obstajata taki realni števili a in b, da velja enakost $p_k = p_{k-1}(a + \frac{b}{k}).$ Dokazati moramo, da je $p$ v Panjerjevem razredu, če je Diracova masa v 0, Poissonova porazdelitev, negativna binomska porazdelitev ali binomska porazdelitev.
\\[1.5mm] Ker mora biti $p_k \geq 0$ za $\forall k \in \mathbb{N} \Rightarrow a+b \geq 0$.
\begin{enumerate}
	
	\item \textbf{$a+b=0$}:\\ $$p_1 = p_0(a+b)=0=p_2=p_3= \ldots =p_k.$$ Ker mora veljati $$1 = \sum_{k=0}^{\infty}p_k,$$sledi $p_0=1$ in tako dobimo Diracovo maso v točki 0.
	
	\item \textbf{$a+b>0$}: 
	\begin{itemize}
		\item $0<a<1$:  Določimo novo spremenljivko $\alpha=\frac{a+b}{a},$ sledi $ b=a(\alpha-1),$ \\ $$p_1=p_0(a+b)=p_0 a \alpha,$$  $$p_2=p_1(a+\frac{b}{2})=p_1a(\frac{\alpha}{2}+\frac{1}{2})=p_0a^2\alpha(\alpha + 1) \frac{1}{2},$$  $$\vdots$$  $$p_k=p_0a^k\frac{1}{k!}\frac{(\alpha+k-1)!}{(\alpha-1)!}=p_0 a^k \binom{\alpha+k-1}{k}.$$ \\\ {Veljati mora $$\sum_{k=0}^{\infty}p_k=\sum_{k=0}^{\infty}\binom{\alpha+k-1}{k}p_0a^k=p_0\sum_{k=0}^{\infty}\binom{\alpha+k-1}{k}a^k=1.$$ Velja tudi: \\ $$\binom{\alpha+k-1}{k}=\frac{(\alpha+k-1)(\alpha+k-2)\ldots(\alpha+k-(k-1))(\alpha+k-k)(\alpha-1)!}{k!(\alpha-1)!}=$$ \\ $$=(-1)^k\frac{(-\alpha)(-\alpha-1)\ldots(-\alpha-k+1)}{k!}=(-1)^k\binom{-\alpha}{k}.$$ \\ Torej je: \\ $$p_0\sum_{k=0}^{\infty}\binom{\alpha+k-1}{k}a^k=p_0\sum_{k=0}^{\infty}\binom{-\alpha}{k}(-1)^ka^k=p_0\sum_{k=0}^{\infty}\binom{-\alpha}{k}(-a)^k=p_0(1+(-a))^{-\alpha},$$ \\ kjer smo upoštevali binomsko vrsto in dejstvo, da je $|a|<1.$ Iz tega sledi enakost $p_0=(1-a)^{\alpha}.$ Od tod dobimo negativno binomsko porazdelitev.}

		\item a = 0: Velja $p_k = p_{k-1} \frac{b}{k}$. Če razpišemo, dobimo:
		$$ \begin{array}{c}
		p_1 = p_0 b, \\[1.5mm]
		p_2 = p_0 \frac{b^2}{2}, \\[1.5mm]
		\ldots \\[1.5mm]
		p_k = p_0 \frac{b^k}{k!}.
		\end{array} $$
		Ker mora veljati \\
$$
		\sum_{k=0}^{n} p_k = 1, \text{ sledi }  \sum_{k=0}^{\infty} p_0 \frac{b^k}{k!} = p_0 \sum_{k=0}^{\infty} \frac{b^k}{k!} = p_0 e^b = 1.$$ Torej velja enakost $p_0 = e^{-b} \text{ in } p_k = e^{-b} \frac{b^k}{k!},$ kar pa je ravno Poissonova porazdelitev s parametrom b.

		\item  a < 0:
			\begin{math}
			\lim_{k \to \infty}
			\frac{p_k}{p_{k-1}} = 
			\lim_{k \to \infty} (a + \frac{b}{k}) = a < 0
			\end{math}, iz česar sledi, da obstaja tak N $\in \mathbb{N}$, da velja $a + \frac{b}{N + 1} = 0$ (vsi členi od nekega N dalje so enaki 0, ker mora biti verjetnostna masna funkcija nenegativna). Če izrazimo $b$, dobimo $b = -a(N+1)$. To vstavimo v Panjerjevo zvezo $p_k = p_{k-1} (a + \frac{b}{k})$:
			$$ \begin{array}{c}
			p_1 = p_0 (a - a(N+1)) = p_0 a (-1) N, \\[1.5mm]
			p_2 = p_1 (a - a \frac{N+1}{2}) = p_0 a^2 \frac{1}{2} (-1)^2 N (N-1), \\[1.5mm]
			\vdots \\[1.5mm]
			p_k = p_0 a^k \frac{1}{k!} N (N-1) \cdots (N - k + 1) (-1)^k\\[1.5mm] = p_0 a^k \frac{1}{k!} \frac{N!}{(N-k)!} (-1)^k = p_0 (-a)^k \binom{N}{k}.
			\end{array} $$
			
Vemo, da mora veljati 
$$
\sum_{k=0}^{\infty} p_k = 1 \rightarrow \sum_{k=0}^{N} \binom{N}{k} p_0 (-a)^k = p_0 \sum_{k=0}^{N} \binom{N}{k} (-a)^k = p_0 (1-a)^N = 1 \rightarrow p_0 = (1-a)^{-N} .$$ (Uporabili smo binomski izrek.) \\[1.5mm]
Če vstavimo $a = \frac{-p}{1-p}$ dobimo: \\[1.5mm]
$p_k = \binom{N}{k} (1+ \frac{p}{1-p})^{-N} (\frac{p}{1-p})^k = \binom{N}{k} (\frac{1}{1-p})^{-N} \frac{p^k}{(1-p)^k} = \binom{N}{k}(1-p)^{N-k} p^k$, \\[1.5mm]
kar pa je ravno binomska porazdelitev.
	 \end{itemize}
\end{enumerate}

\end{document}
